\chapter{Partial Solutions}
\section{Identity and secrets}
\subsection{Public-key cryptography}
We often refer to the ``identity'' of an agent or network participant. We use this term to refer to the keypair that a node generates, and by which it identifies itself. Such a keypair consists of a public and private part: the first publicly known, the latter secret. Those in possession of a private key can demonstrate its ownership by \emph{signing} pieces of data. Others can then use the public part of the key to verify the correctness of the signature. The keypair can also be used for encryption: all who know a public key can use it to encrypt a piece of information in such a way that only the holder of the corresponding private key can decrypt it.

\subsection{Atomic transactions}
One of the many assumptions made in section \ref{section:model_notation} was the atomicity of transactions. 
This appears trivial at first, but is not in practice. 
How should one transfer value at the same time as one records this value transfer? 
A transaction participant could refuse to perform work after a transaction has been signed, 
or an actor receiving work could refuse to sign a transaction after the work has been completed.

\subsubsection{A simple scheme}
For this reason we provide an example of a simple transaction scheme, 
for which it is beneficial for participant, not to cheat. 
The protocol consists of several steps. 
Participant $A$ receives and pays for work performed by $B$.
\begin{enumerate}
	\item $A$ provides cryptographic proof to $B$ demonstrating it agrees to a price for a specific good
	\item \label{atomic_transactions_enumerate_proof_of_work}$B$ performs work and \emph{obtains a cryptographic proof of work performed}
	\item $B$ publishes proof of work completed
	\item Now either 
	\begin{enumerate}
		\item $A$ and $B$ sign a transaction and complete the procedure
		\item $A$ refuses to "pay" $B$ by signing the transaction, 
		and $B$ proves to the network that $A$ cheated
	\end{enumerate}
\end{enumerate}
Naturally, being proven to be a cheater should negatively impact reputation. 
Note how $B$ can't blame $A$ without both the proof of $A$ willing to receive work, 
and the proof of $B$ having performed the work.

At step \ref{atomic_transactions_enumerate_proof_of_work} we skipped over how to exactly provide cryptographic proof of work performed. 
This type of proof should depend on the application. 
It should however always be at least as hard to obtain the proof as to perform the actual work.

\subsubsection{Proof of transmission}\label{sssection:proof_of_transmission}
In the case of file transfers, 
it is impossible to cryptographically prove that data has been sent over a passive channel \cite{kravchenko}. 
It is however possible for an agent to prove to another that it is in possession of the data to be sent. 
This is called a "proof of custody", which is essentially a zero-knowledge proof which can only be performed by an agent in possession of a private key.

\section{Double-spend and fork prevention}
An agent $A$ might try to benefit by using its current chain to compel another agent $B$ to make a (large) transaction. After the transaction has been made, $A$ can hide the transaction made with $B$ from $C$, with whom he then does a second (large) transaction which $C$ would not have done if it had know about the first transaction. This is also known as "double spending".

In order to prevent an agent from making a transaction while withholding previous transactions, those agents collecting blocks should always look for two blocks made by the same agent pointing to the same block: this implies a double-spend. Once a double-spend is detected, and the author of the transaction is marked as a ``cheater'', agents should refuse to transact with this person again.

\section{Sybil prevention}\index{Sybil attack!Prevention}
\subsection{Fundamental Limitations}
It can be shown that Sybil attacks cannot always be prevented. 
For instance, when using a reputation function that only depends on the topology of the network graph, 
it can be shown that such a system is always susceptible to Sybil attacks \cite{cheng2005sybilproof}. 
This means relative positions, 
for instance compared to some trusted user, 
need to be used. 
Even in this case, 
there are limitations to preventing Sybil attacks. 
Nevertheless, it might suffice to make Sybil attacks unprofitable, 
instead of completely eliminating the possibility of such an attack. 

\subsection{Proof-of-work system for identity creation}\label{sssection:chain_creation_cost}
While generating a keypair is cheap, this can be made more costly by imposing certain requirements on the keypair. For example, a protocol could require all public keys to have a certain low hash value.

Depending on several factors, this measure can be effective. This measure is limited by the low cost with which honest users should be able to create a keypair. A reasonable requirement for keypair generation could be ``within three seconds on a smartphone''. In that case, a determined attacker can still easily generate an enormous number of identities with general-purpose computing on graphics processing units.

\subsection{Tying identities to IP-addresses}
As suggested in \cite{meulpolder2009bartercast}, it is possible to link identities to IP addresses, allowing only a fixed number of identities to exist for every IP address.

\subsection{Latency}
By requiring a limited link latency for all agents interacted with, participants can force the geographical proximity of their transaction partners. Otherwise they could ask another trusted agent to test their geographical proximity to the agent being considered for a transaction.

\subsection{NetFlow}\index{Sybil attack!NetFlow}
When sybils are being used by attackers to feign contribution to the network, 
there is typically much traffic between the attacker and its pseudonymous identities, 
and less across the network. 
Therefore, the NetFlow algorithm uses max-flow computations to determine the traffic flow across the whole network, 
serving as a metric for trustworthiness of agents in the network. Hence, the profit of cheating using Sybil effects is bounded. 
However, the NetFlow algorithm is computationally expensive, 
which means it is not fast enough for most practical applications. 
The algorithm could possibly be reconsidered, 
or replaced by a (probabilistic) approximation variant. 

\section{Consensus}
\subsection{Checo}

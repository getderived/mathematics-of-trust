\chapter{Further research}
We identify unanswered questions several area's of future research. 
This chapter provides an overview of these questions and attempts to structure them.
Note that barely any of the differences between the theory and practice as highlighted in chapter \ref{chapter:differences} are solved problems, let alone the interplay between them.

\section{Identity and secrets}
\subsection{Derived identities}
If one is willing to accept a third, central party as a trusted distributor of identities only, such as a government, it may be possible to address the sybil attack problem as follows. Use the identity (public-private keypair) as received from the trusted third party to create a derived identity for a specific application. We now need a zero-knowledge proof scheme to show that any derived identity is backed by a master identity, which can't be produced in large quantities. Moreover, a zero-knowledge proof is needed to demonstrate that two derived identities are not derived from the same master identity, such that for any specific application, only a single derived identity can be created by each master identity. In this way, users can be anonymous while be certain that no individual has multiple identities.

\subsection{Proof of transmission}
In section \ref{sssection:proof_of_transmission} it was mentioned that a proof of transmission over a \emph{passive} channel is impossible. 
A passive channel is one in which ``does not modify or sign the data it transmits'' \cite{kravchenko}. 
It may however be possible to realize such a proof using other nodes in the network.

\section{Sybil prevention}
\subsection{Computational feasibility of NetFlow}
While NetFlow gives strong guarantees for sybil resistance, the computational complexity of the algorithm is too great \cite{otte2016sybil}. NetFlow could, for example,
\begin{itemize}
	\item be approximated, while its sybil defense mechanism deteriorates only an acceptable amount, or
	\item be extended with the exchange of results of computations.
\end{itemize}
% TODO: Expand upon this

\subsection{PageRank}
While PageRank is relatively cheap to approximate precisely, it is unclear what guarantees it provides, if any. Among the open questions about PageRank are:
\begin{itemize}
	\item What are the different metrics we can calculate the PageRank score on, and what do they tell us about the participants?
	\item Does the PageRank score provide any (probabilistic) guarantees?
	\item Should temporal aspects be included in this computation? How?
	\item How could different PageRank scores, computed on different network metrics, complement each other?
\end{itemize}
% TODO: Expand upon this

\subsection{About time}
An open question is how to prioritize older transactions made. Should they be removed from computations, made less important, or be regarded just as important as the most recent transactions? This is interesting from both a normative point of view, as well as a utilitarian point of view considering changes in effectiveness of for example sybil prevention.

\section{Advanced concepts}
\subsection{Smart contracts}
It's interesting to consider what an execution model for a DAG blockchain or tangle would look like. The triggering of logical rules using state change can't be synchronized globally, as this would require global consensus. Instead, alternative models need to be defined.

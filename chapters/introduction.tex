\chapter{Introduction}
\section{Historical Perspective}
Cooperation is a hard problem. 
In many environments, self-interested and rational choice making results in a suboptimal outcome for all. 
Arguably the best known example of this problem is the prisoner's dilemma, 
where defection is a dominant strategy. 
The prisoners' inability to coordinate and build expectations of the choice made by the other prisoner ensures that both will be worse off than if they had collaborated.

In real life, we encounter more often a repeated version of the prisoner's dilemma, or the iterated prisoner's dilemma. 
We often meet the same people. 
For this game, different strategies exist. 
% TODO: Introduce to Tragedy of the Commons

\subsection{Trust}
There are various definitions of trust, 
but all have a few basics facets to them. 
Wikipedia defines trust as a situation characterized by the following aspects \cite{wikipedia_trust}:
\begin{itemize}
	\item one party (trustor) is willing to rely on the actions of another party (trustee); 
	the situation is directed to the future
	\item the trustor (voluntarily or forcedly) abandons control over the actions performed by the trustee
	\item the trustor is uncertain about the outcome of the other's actions; 
	they can only develop and evaluate expectations
	\item the uncertainty involves the risk of failure or harm to the trustor if the trustee will not behave as desired
\end{itemize}
We can view trust as necessary catalyst for practical collaboration.
Trust systems have existed ever since people lived in close-knit, local societies. Mutual dependence for survival, a lack of movement and gossip were sufficient to build small systems of trust.
In the industrial age, large companies and brands became commonplace.
The large investment in brand recognition is can be viewed as a ``costly-to-fake'' signal, leading consumers to trust in the companies intention to sustain its brand value, and sell high-quality products.

More recently, a new type of trust system emerged. Companies like eBay and AirBnB allow consumers to interact directly without having prior knowledge of each other. A ``two-way ranking'' system is facilitated by these companies, where any two consumers interacting via the platform give a review of their transaction partner. These companies then capitalize on the trust that consumers have in each other.

We want to realize trust systems which allow individuals which have never interacted before, to trust each other without any central entity involved. Blockchains are the first successful attempt at realizing such a system. 

\section{Blockchains}
Blockchains are data structures which utilize cryptographic primitives such as public-key cryptography and trapdoor functions. The most well-known blockchain is arguably Bitcoin, the first implementation of a blockchain used for financial transactions. The currency uses a blockchain as a ledger to store past transactions. These transactions are grouped in so-called ``blocks''. The ledger is composed of a sequence such blocks; they form the ledger. All blocks point to the previous one by containing the a cryptographic hash of the previous block. Blocks can only be appended to the chain, and are immutable; modification of a block changes its hash and in this way ``breaks'' the chain.

\subsection{Consensus in proof-of-work blockchains}
All major blockchains currently use ``proof-of-work'' as a fraud prevention mechanism. New blocks are formed by ``miners'', nodes in the network which collect transactions, group them with the hash of the previous block. They repeatedly append a random nonce to the data and calculate the cryptographic hash, until they find a nonce which results in a cryptographic hash satisfying certain properties. Once such a nonce is found, the transactions, the previous block's hash and the nonce are published to the network; they form the new block. 

The miner which ``finds'' a new block is awarded with some currency in the ledger. Sometimes, two blocks are found closely after one another. Then, the longest chain determines the latest state of the network. The proof-of-work mechanism ensures that mining is costly, and that no single entity can control which chain is the longest by producing many blocks.

In all major blockchains, consensus on the latest state of the ledger is formed by mining a block. This requires that all transactions are globally known, which fundamentally limits the scalability of such a system. In pursuing a more scalable solution, another data structure has been invented called ``TrustChain''.

\section{TrustChain}
In TrustChain, all network participants have maintain their own chain. There is no mining, and no global consensus. TrustChain works fundamentally different than existing blockchains; we sometimes call these blockchains ``fourth-generation blockchains''.

Users create a new chain by generating a public-private keypair and a first, empty block. When they participate in a transaction with another TrustChain user, they create a block pointing to both their own, and their transaction partner's last block. Next to the transaction contents, they append a signature using they private key. An example of blocks and references is given in Figure \ref{fig:tangle_example}.

\begin{figure}[h]
	\centering
	\begin{tikzpicture}[->,thick,scale=0.4,main node/.style={circle, draw, fill=white,
		inner sep=0pt, minimum width=20pt} 
	]
	
	\node [main node] (p1) {P1} ;
	\node [main node] (p2) [below = 1.5cm of p1] {P2} ;
	\node [main node] (q1) [right = 3cm of p1] {Q1} ;
	\node [main node] (r1) [right = 1.5cm of p2]  {R1} ;
	\node [main node] (r2) [below = 1.5cm of r1] {R2} ;
	\node [main node] (q2) [below = 1.5cm of q1] {Q2} ;
	\node [main node] (r3) [below = 1.5cm of r2] {R3} ;
	\node [main node] (p3) [below = 1.5cm of p2] {P3} ;
	\node [main node] (r4) [below = 1.5cm of r3] {R4} ;
	\node [main node] (p4) [below = 1.5cm of p3] {P4} ;
	\node [main node] (q3) [below = 1.5cm of q2] {Q3} ;
	
	
	
	\path[draw, thick]
	(p1) edge node [right] {} (p2)
	(p2) edge node [left] {} (p3)
	(q1) edge node [left] {} (q2)
	(r1) edge node [right] {} (r2)
	(r2) edge node [right] {} (r3)
	(p1) edge node [above right, pos=0.1] {} (q2)
	(q1) edge node [above left, pos=0.1] {} (p2)
	(p2) edge node [above right, pos=0.1] {} (r2)
	(r1) edge node [above left, pos=0.1] {} (p3)
	(q2) edge [bend left=60] node [right] {} (r3)
	(p3) edge node [left] {} (p4)
	(r3) edge node [above, pos=0.1] {} (p4)
	(r3) edge node [left] {} (r4)
	(p3) edge node [right, pos=0.1] {} (r4)
	(q2) edge node [left] {} (q3)
	(r2) edge node [above] {} (q3)
	
	
	;
	\end{tikzpicture}
	\caption{An example of a collection of blocks and pointers. Adapted from \cite{otte2016sybil}.} \label{fig:tangle_example}
\end{figure}

Users share (part of) their chains with the people they transact with, and possibly share their blocks with and collect blocks from others. This data structure is extremely scalable. But because there is no global consensus, several problems arise. Before we analyze possible solutions, we will start from a model which attempts to capture the motivations of those participating in the network.

\chapter{Requirements}
\section{Decentralization}
Naturally, the system should be completely decentralized. It is especially the cooperation independent of a central party we want to realize.

\section{Scalability}
A universal mechanism to create trust requires it being usable by the entire world with absolute minimal limitations of usage. For this reason, we require of any system which attempts to solve the trust problem that it is scalable in three ways. The first being in the number of users that can participate in the system freely, concurrently and without interruption. Secondly, we need users to be able to make a large number of transactions in short periods of time, always. The third scalability concerns time: as the system gets used for a longer period of time, the previous two scalability requirements shouldn't become insurmountable to achieve.

\section{Not fragile and truly peer-to-peer}
All those in the network should perform the same set of tasks. The appointment of special roles often results in fragile systems which are (temporarily) dependent on a very small number of participants. A single node type results in simpler, more resilient designs.

\section{No ``early adopter takes all''}
Many cryptocurrencies know an extremely small group of early adopters which own an large portion of all coins. We aim to create the possibility to trust strangers, rather than redistributing wealth.
